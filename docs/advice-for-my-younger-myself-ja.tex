\documentclass[a4paper,11pt,uplatex]{jsarticle}
\usepackage{graphicx}
\usepackage{enumitem}
\usepackage[dvipdfmx]{hyperref}

\title{プログラミングを始めたばかりの過去の自分へ}
\author{}
\date{}

\begin{document}

\maketitle

\section*{あなたは数年後にこんな感じで遊び始めます}
\href{https://photos.onedrive.com/share/12F1169924695EF9!250373?cid=12F1169924695EF9&resId=12F1169924695EF9!250373&authkey=!AEcanzCk1fx4cYI&ithint=video&e=TjlgN1}{動画 OneDrive link; 40 分}

\section{基本編}
\begin{enumerate}
    \item Google 検索を極めなさい
    \begin{itemize}
        \item あなたが直面している課題は、誰かがすでに解決しているかもしれません
    \end{itemize}
    \item AI を積極的に活用しなさい
    \item キーボードショートカットを学びなさい
    \begin{itemize}
        \item 壁を印刷したチートシートで埋めなさい
        \item Ctrl + C の便利さを拡張しましょう
    \end{itemize}
\end{enumerate}

\section{マニアック編 (プログラミングを仕事にしたい場合)}

\subsection{心がけ編}
\begin{enumerate}
    \item コンピュータの操作を英語で行いなさい
    \begin{itemize}
        \item コンピュータは第一に日本人のために作られているわけではなさそうです
        \item IELTS で国際的な機会を得ることができます \href{https://ai-ielts.app/}{私の IELTS 学習アプリ}
        \begin{itemize}
            \item 現在カナダとオーストラリアは博士号と IELTS 7.0 以上ほどで永住権が取得できるかもしれません
        \end{itemize}
    \end{itemize}
    \item マウスの使用を最小限に抑えなさい
    \begin{itemize}
        \item なるべくキーボードのホームポジションから手を離さないでください
    \end{itemize}
    \item 技術書、ドキュメント、ブログ記事を読みなさい
    \begin{itemize}
        \item 特に初期段階では、信頼のある技術書で大枠を把握することが役立ちます
    \end{itemize}
\end{enumerate}

\subsection{ソフトウェア編}
\begin{enumerate}
    \item プログラミングには Linux を活用しなさい
    \begin{itemize}
        \item この世界では避けて通ることはできません
        \item \href{https://www.amazon.co.jp/s?k=linux+\%E3\%83\%AC\%E3\%83\%99\%E3\%83\%AB1&crid=RJR3JEEZ0J2T&sprefix=linux+\%2Caps\%2C287&ref=nb_sb_ss_ts-doa-p_7_6}{Linux教科書 LinuCレベル1} の 3 冊を取り敢えず読めば良いと思います
        \item Linux の設定を継続的に改善しましょう \href{https://github.com/ywatanabe1989/.dotfiles-public}{私の dotfiles リポジトリ}
    \end{itemize}
    \item 関数をたくさん作りなさい
    \begin{itemize}
        \item 自動販売機のようなものです、ボタンを押せばコーラが出る、毎回毎回コーラが出る、自販機の中がどうなっていたか気にしなくて済みます
    \end{itemize}
    \item Emacs を学び、使用しなさい
    \begin{itemize}
        \item 私が知らないだけで Emacs(宗教です) でなくてもよいです、キーボードから手を離したくない、引きこもりコードに集中したい、という精神です
        \item チュートリアルから始めなさい \texttt{M-x help-with-tutorial}
        \begin{itemize}
            \item Emacsを起動し、\texttt{M-x help-with-tutorial RET} (Meta (Alt or Esc) キー + x、その後 'help-with-tutorial' と入力し Enter)
        \end{itemize}
        \item \href{https://www.youtube.com/playlist?list=PLEoMzSkcN8oPH1au7H6B7bBJ4ZO7BXjSZ}{SystemCrafters YouTube チャンネル}
        \begin{itemize}
            \item 流して見るだけで楽しいです
        \end{itemize}
        \item \href{https://github.com/ywatanabe1989/.dotfiles-public/tree/main/.emacs.d/inits}{私の Emacs 設定}
        \begin{itemize}
            \item 自分の設定を一から作ると何が起こっているのか、どのようにプログラミングしたら良いのか、理解出来ると思います
        \end{itemize}
    \end{itemize}
    \item 個人的なツールボックスを更新し続けなさい
    \begin{itemize}
        \item 動く関数が引き出しのどこかにあるとは役に立ちます \href{https://github.com/ywatanabe1989/mngs}{私の Python ツールキット例}
    \end{itemize}
    \item 独自のショートカットを作成し、最適化しなさい
    \begin{itemize}
        \item \href{https://github.com/ywatanabe1989/.dotfiles-public/tree/main/.bash.d/all}{私の bash ショートカット}
        \item \href{https://github.com/ywatanabe1989/.dotfiles-public/.emacs.d/}{私の Emacs ショートカット}
    \end{itemize}
    \item GitHub を活用しなさい
    \begin{itemize}
        \item 他人のために書きなさい。他人とは将来のあなたであり、他のマシン上でのあなたです。\href{https://github.com/ywatanabe1989/}{私の GitHub レポジトリ}
    \end{itemize}
\end{enumerate}

\subsection{道具編}
\begin{enumerate}
    \item \href{https://hhkeyboard.us/hhkb/pro-hybrid-type-s/sku/cg01000-297301}{無刻印の HHKB キーボード}を使用しなさい
    \begin{itemize}
        \item 決してキーボードを見ることがなくなるので(意味がないので)、どんなに細かい記号や修飾キーの組み合わせでも、感覚で覚えるようになります
        \item 例えば左の手の小指の付け根で左下の Fn キーを押したりします
    \end{itemize}
    \item 一度はパソコンを組みなさい
    \begin{itemize}
        \item パーツを理解するためにとても有効です
        \item 電源、マザーボード、CPU、CPU クーラ、ストレージ、GPU、キーボード、マウスぐらいです
    \end{itemize}
\end{enumerate}

\section{おわりに}
コーディングを楽しんでください

\end{document}
